\chapter{Introduction}
%http://eetd.lbl.gov/ee/ee-3.html - who does efficiency?
% Tell policies and other over all big picture
%http://spectrum.ieee.org/computing/hardware/moores-law-might-be-slowing-down-but-not-energy-efficiency
%http://spectrum.ieee.org/tech-talk/computing/hardware/mit-researchers-build-a-100fold-more-efficient-transmitter-
%http://whatis.techtarget.com/definition/Internet-of-Things
%http://www.slideshare.net/IoTMethodology/a-methodology-for-building-the-internet-of-things-42112202?ref=http://www.iotmethodology.com/
%\subsection{The Context of Interest: Smartphones}
IoT is a convergence of number of technologies such as \textit{sensors}, \textit{IPv6}, \textit{wireless communication} and  \textit{Internet}. Any real-world objects become smart just by satisfying few conditions but not limited to: 1) uniquely identifiable;  2) being able to sense or actuate; 3) being able to communicate \cite{IoT:Defn}. The growth of smart objects are posing challenges to the research community in energy management, data analytics and security \cite{IoT:Challenge}. Among these challenges, security and privacy issues are not just issues in technical system design level, but also in ethical, behavioural and policies level. We have powerful analytical tools available with advanced data analysis algorithms \cite{BigD:Deep}. On the other hand, energy management is more complex and chaotic, that is our focus of this paper. 

%\subsection{Energy Efficiency: The Ins and Outs}
Berkeley National Laboratory defined energy efficiency as using less energy to provide the same service. The need for energy efficiency highly inevitable in almost every type of industries, companies and organizations including Information and Communications Technology (ICT). Energy management in Internet of Things(IoT) aims at reducing the electricity, which is beneficial for many industries to reduce their electricity bills. As the smart objects becoming smaller in size, their small sized batteries provide limited power for operations. Even the smart appliances are idle, they could indirectly waste huge amount of energy in long term and eventually increase the electricity bills too. "Although ICT can enable energy efficiency across all sectors, at present there is little market incentive to ensure that network-enabled devices themselves are energy efficient. In fact, up to 80\% of their electricity consumption is used just to maintain a network connection. While the quantity of electricity used by each device is small, the anticipated massive deployment and widespread use makes the cumulative consumption considerable" as reported by International Energy Agency in \cite{IEA:bdle}.

%\subsection{Energy of Energy Efficiency}
Hereafter we narrow our focus on smart-phones which are increasing in exponential order over the years. Modern smart-phones provide heterogeneous functionalities including a number of sensors. They are one of the most representative and popular smart objects in the IoT. As smart-phones are resource constraint with respect to memory and computation, they happen to off-load computation and access remote storage on the cloud servers via network. Cloud computing in the IoT leads to thousands of cloud supported applications and it is growing steeply. As a consequence, smart-phones are consuming a lot of energy for communication with the cloud. Due to the size limitation, effort of making powerful batteries is not able to withstand the energy hungriness persist in the smart-phones. It is important to reduce energy consumption when developing new kind of applications.

Smart-phones are usually running multiple applications with different operations at the same time. It is very difficult to understand and identify the cause of high energy consumption in this asynchronous power consuming environment. It is necessary to provide profiling of power consumption in at different levels, such as whole system, individual applications, and system calls in operation level. In this paper, we propose the first iterative novel solution using \textit{Cloud Orchestration} for power management on smart-phones. Cloud orchestration aggregates power profiling data from the smart-phones and coordinates data storage, data analysis, learning and decision making. From the profiling data, the orchestrator learns mainly the power consumption behaviours and the usage pattern of the participating smart-phones. The orchestrator aims for providing overall system power management rather than making part of the system efficient. 


