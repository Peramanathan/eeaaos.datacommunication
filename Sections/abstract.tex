This paper proposes a novel power management solution for the resource-constrained devices in the context of Internet of Things (IoT). We focus on smart-phones in the IoT, as they are getting increasingly popular and equipped with strong sensing capabilities. Smart-phones have complex and chaotic asynchronous power consumption incurred by heterogeneous components including their on-board sensors. Their interaction with the cloud can support computation offloading and remote data access via the network. In this work, we aim at monitoring the power consumption behaviours of smart-phones and profiling individual applications and platform to make better decisions in power management. A solution is to design architecture of cloud orchestration as an epic predictor of the behaviours of smart devices with respect to time, location, and context. We design and implement this architecture to provide an integrated cloud-based energy monitoring service. This service enables the monitoring of power consumption on smart-phones and support data analysis on massive data logs collected by large number of users.





%This paper proposes a naval energy management solution for the resource constrained systems in the context of Internet of Things(IoT). Here smartphones are chosen as the representative systems. They have superlative complex and chaotic asynchronous power consumption incur by the heterogeneous components including number of sensors and by triggering activities including off-loading computation, accessing remote cloud storage via network data communication. A solution is  to design architecture of cloud orchestration--a epic predictor of the participatory devices behaviours with respect to time, location, in general and any needed arbitrary context--and investigating towards successful implementation of the service. Upon enabled to the service the local infant application in participatory devices fed sufficiently by the orchestrator over the course of time. Thus by act themselves energy efficient and reach only needed not to overwhelm data communication.