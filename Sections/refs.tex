\bibitem{android:evol}
Chandnani, P., \& Wadhvani, R. . “Evolution of Android and its Impact on Mobile Application Development. International Journal of Scientific Engineering and Technology”  1(1), 80–85 ,  2012.


\bibitem{IoT:Defn}
J. Gubbi, R. Buyya, S. Marusic, and M. Palaniswami, “Internet of Things (IoT): A vision, architectural elements, and future directions,” Future Generation Computer Systems, vol. 29, no. 7, pp. 1645–1660, Sep. 2013.
\bibitem{IoT:Challenge}
J. Chase, The Evolution of the Internet of Things. Dallas, TX: Texas Instruments Incorporated, 2013.
\bibitem{BigD:Deep}
M. M. Najafabadi, F. Villanustre, T. M. Khoshgoftaar, N. Seliya, R. Wald, and E. Muharemagic, “Deep learning applications and challenges in big data analytics,” Journal of Big Data, vol. 2, no. 1, Feb. 2015.

\bibitem{IEA:bdle}
International Energy Agency, "More Data, Less Energy - Making Network Standby More Efficient in Billions of Connected Devices" \copyright OECD/IEA, 2014 Licence:\url{https://www.iea.org/t&c/termsandconditions/}

\bibitem{EE:MCCarch}
A. Tzanakaki, M. P. Anastasopoulos, S. Peng, B. Rofoee, Y. Yan, D. Simeonidou, G. Landi, G. Bernini, N. Ciulli, J. F. Riera, and others, “A converged network architecture for energy efficient mobile cloud computing,” in Optical Network Design and Modeling, 2014 International Conference on, 2014, pp. 120–125.

\bibitem{EE:JVM}
T. K. Kundu and K. Paul, “Improving Android Performance and Energy Efficiency,” 2011, pp. 256–261.

\bibitem{EE:eTime}
P. Shu, F. Liu, H. Jin, M. Chen, F. Wen, and Y. Qu, “eTime: energy-efficient transmission between cloud and mobile devices,” in INFOCOM, 2013 Proceedings IEEE, 2013, pp. 195–199.

\bibitem{EE:Multinets}
S. Nirjon, A. Nicoara, C.-H. Hsu, J. Singh, and J. Stankovic, “Multinets: Policy oriented real-time switching of wireless interfaces on mobile devices,” in Real-Time and Embedded Technology and Applications Symposium (RTAS), 2012 IEEE 18th, 2012, pp. 251–260.

\bibitem{EE:Hier}
J. Tang, Z. Zhou, J. Niu, and Q. Wang, “An energy efficient hierarchical clustering index tree for facilitating time-correlated region queries in the Internet of Things,” Journal of Network and Computer Applications, vol. 40, pp. 1–11, Apr. 2014.

\bibitem{EE:Protocol}
A. Venčkauskas, N. Jusas, E. Kazanavičius, and V. Štuikys, “An Energy Efficient Protocol For The Internet Of Things,” Journal of Electrical Engineering, vol. 66, no. 1, pp. 47–52, 2015.

\bibitem{EE:GreenSW}
B. Steigerwald and A. Agrawal, “Developing green software,” Intel White Paper, vol. 9, 2011.

\bibitem{EE:Measure}
A. Pathak, Y. C. Hu, and M. Zhang, “Where is the energy spent inside my app?: fine grained energy accounting on smartphones with eprof,” in Proceedings of the 7th ACM european conference on Computer Systems, 2012, pp. 29–42.

\bibitem{IoTA:ARM}
R. Beneficiary, I. M. L. FhG, S. H. SAP, E. H. HSG, C. Jardak, A. O. CEA, A. Serbanati, M. T. SAP, and J. W. Walewski, “Internet of Things-Architecture IoT-A Deliverable D1. 3–Updated reference model for IoT v1. 5.”
\bibitem{Orches:Arch}
A. González García, M. Alvarez Alvarez, J. Pascual Espada, O. Sanjuán Martínez, J. M. Cueva Lovelle, and C. Pelayo G-Bustelo, “Introduction to Devices Orchestration in Internet of Things Using SBPMN,” International Journal of Interactive Multimedia and Artificial Intelligence, vol. 1, no. 4, p. 16, 2011.

\bibitem{IoT:Survey}
L. Atzori, A. Iera, and G. Morabito, “The Internet of Things: A survey,” Computer Networks, vol. 54, no. 15, pp. 2787–2805, Oct. 2010.
\bibitem{IoT:Arch}
H. Suo, J. Wan, C. Zou, and J. Liu, “Security in the Internet of Things: A Review,” 2012, pp. 648–651.
\bibitem{EModel}
A. Pathak, Y. C. Hu, M. Zhang, P. Bahl, and Y.-M. Wang, “Fine-grained power modeling for smartphones using system call tracing,” in Proceedings of the sixth conference on Computer systems, 2011, pp. 153–168.
\bibitem{trepn}
Qualcomm Technologies, Inc. "Trepn Profiler" \url{https://developer.qualcomm.com/mobile-development/increase-app-performance/trepn-profiler}, Apr. 15, 2015.