\chapter{Conclusions}
\section{Requirements}
\section{Issues}
\section{Future Work}
Orchestrator is in essence the behaviour predictor of the participating devices with respect to time, location and as many as added contexts. The agent application installed in the smart-phones which report logs to the orchestrator will be facilitated with local validator and action triggers which is regularly updated by orchestrator according to the needs to avoid wasting orchestrator energy and data communication for well learnt case. In order to successfully deploy such orchestration service, we need to study and explore all its components defined in the previous section and their interdependencies in detail.  Then we focus on the questions including 1) how to develop low energy consuming profiler? 2) how to reduce logs reporting thus by reduce data communication? 3) how to make orchestrator an epic predictor of device behaviours? 4) how to find optimal responsibilities of local agent by ensuring minimal computation ans resources? 5) Is it best fit for mass open source contribution? 6) What are the de-facto tools for over all implementation? We plan to implement and test the prototype iteratively. This work could be then extended or simplified to other type of IoT devices.
\section{Conclusion}
In this paper we proposed a novel solution for improving energy efficiency in smartphones as a cloud orchestration. We have explained the components of the orchestration and their functionalities. The architecture design is flexible so that new components can be added easily. The \emph{big data}, \emph{knowledge graph}, \emph{control box} will be accessible openly so that mass collaborators can participate and test. It will help improving the orchestrator implicitly or at the least for bug reporting. There is a potential that the big data knowledge would be useful in solving a number of other problems and enabling new services.